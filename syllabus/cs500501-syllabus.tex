% CS 580 style
% Typical usage (all UPPERCASE items are optional):
%       \input 580pre
%       \begin{document}
%       \MYTITLE{Title of document, e.g., Lab 1\\Due ...}
%       \MYHEADERS{short title}{other running head, e.g., due date}
%       \PURPOSE{Description of purpose}
%       \SUMMARY{Very short overview of assignment}
%       \DETAILS{Detailed description}
%         \SUBHEAD{if needed} ...
%         \SUBHEAD{if needed} ...
%          ...
%       \HANDIN{What to hand in and how}
%       \begin{checklist}
%       \item ...
%       \end{checklist}
% There is no need to include a "\documentstyle."
% However, there should be an "\end{document}."
%
%===========================================================
\documentclass[11pt,twoside,titlepage]{article}
%%NEED TO ADD epsf!!
\usepackage{threeparttop}
\usepackage{graphicx}
\usepackage{latexsym}
\usepackage{color}
\usepackage{listings}
\usepackage{fancyvrb}
%\usepackage{pgf,pgfarrows,pgfnodes,pgfautomata,pgfheaps,pgfshade}
\usepackage{tikz}
\usepackage[normalem]{ulem}
\tikzset{
    %Define standard arrow tip
%    >=stealth',
    %Define style for boxes
    oval/.style={
           rectangle,
           rounded corners,
           draw=black, very thick,
           text width=6.5em,
           minimum height=2em,
           text centered},
    % Define arrow style
    arr/.style={
           ->,
           thick,
           shorten <=2pt,
           shorten >=2pt,}
}
\usepackage[noend]{algorithmic}
\usepackage[noend]{algorithm}
\newcommand{\bfor}{{\bf for\ }}
\newcommand{\bthen}{{\bf then\ }}
\newcommand{\bwhile}{{\bf while\ }}
\newcommand{\btrue}{{\bf true\ }}
\newcommand{\bfalse}{{\bf false\ }}
\newcommand{\bto}{{\bf to\ }}
\newcommand{\bdo}{{\bf do\ }}
\newcommand{\bif}{{\bf if\ }}
\newcommand{\belse}{{\bf else\ }}
\newcommand{\band}{{\bf and\ }}
\newcommand{\breturn}{{\bf return\ }}
\newcommand{\mod}{{\rm mod}}
\renewcommand{\algorithmiccomment}[1]{$\rhd$ #1}
\newenvironment{checklist}{\par\noindent\hspace{-.25in}{\bf Checklist:}\renewcommand{\labelitemi}{$\Box$}%
\begin{itemize}}{\end{itemize}}
\pagestyle{threepartheadings}
\usepackage{url}
\usepackage{wrapfig}
% removing the standard hyperref to avoid the horrible boxes
%\usepackage{hyperref}
\usepackage[hidelinks]{hyperref}
% added in the dtklogos for the bibtex formatting
\usepackage{dtklogos}
%=========================
% One-inch margins everywhere
%=========================
\setlength{\topmargin}{0in}
\setlength{\textheight}{8.5in}
\setlength{\oddsidemargin}{0in}
\setlength{\evensidemargin}{0in}
\setlength{\textwidth}{6.5in}
%===============================
%===============================
% Macro for document title:
%===============================
\newcommand{\MYTITLE}[1]%
   {\begin{center}
     \begin{center}
     \bf
     CMPSC 500 and 501\\Internship Seminar\\
     Fall 2013
     \medskip
     \end{center}
     \bf
     #1
     \end{center}
}
%================================
% Macro for headings:
%================================
\newcommand{\MYHEADERS}[2]%
   {\lhead{#1}
    \rhead{#2}
    %\immediate\write16{}
    %\immediate\write16{DATE OF HANDOUT?}
    %\read16 to \dateofhandout
    \def \dateofhandout {}
    \lfoot{}
    %\immediate\write16{}
    %\immediate\write16{HANDOUT NUMBER?}
    %\read16 to\handoutnum
    \def \handoutnum {1}
    \rfoot{Handout \handoutnum}
   }

%================================
% Macro for bold italic:
%================================
\newcommand{\bit}[1]{{\textit{\textbf{#1}}}}

%=========================
% Non-zero paragraph skips.
%=========================
\setlength{\parskip}{1ex}

%=========================
% Create various environments:
%=========================
\newcommand{\PURPOSE}{\par\noindent\hspace{-.25in}{\bf Purpose:\ }}
\newcommand{\SUMMARY}{\par\noindent\hspace{-.25in}{\bf Summary:\ }}
\newcommand{\DETAILS}{\par\noindent\hspace{-.25in}{\bf Details:\ }}
\newcommand{\HANDIN}{\par\noindent\hspace{-.25in}{\bf Hand in:\ }}
\newcommand{\SUBHEAD}[1]{\bigskip\par\noindent\hspace{-.1in}{\sc #1}\\}
%\newenvironment{CHECKLIST}{\begin{itemize}}{\end{itemize}}


\usepackage[compact]{titlesec}

\begin{document}
\MYTITLE{Syllabus}
\MYHEADERS{Syllabus}{}

\subsection*{Course Instructor}
Dr.\ Gregory M.\ Kapfhammer\\
\noindent Office Location: Alden Hall 108 \\
\noindent Office Phone: +1 814-332-2880 \\
\noindent Email: \url{gkapfham@allegheny.edu} \\
\noindent Twitter: \url{@GregKapfhammer} \\
\noindent Web Site: \url{http://www.cs.allegheny.edu/sites/gkapfham/} 

\subsection*{Instructor's Office Hours}

\begin{itemize}
	\itemsep 0em
	\item Monday: 1:00 pm -- 2:30 pm (30 minute time slots)
	\item Tuesday: 2:30 pm -- 4:00 pm (15 minute time slots)
	\item Wednesday: 4:30 pm -- 5:00 pm (15 minute time slots)
	\item Thursday: 9:00 am -- 10:00 am (15 minute time slots) {\em and} \\ \hspace*{.69in} 2:30 pm -- 4:00 pm (15 minute time slots)
	\item Friday: 1:00 pm -- 2:30 pm (10 minute time slots) {\em and} \\ \hspace*{.49in} 4:30 pm -- 5:00 pm (5 minute time slots)
\end{itemize}

\noindent
To schedule a meeting with me during my office hours, please visit my Web site and click the ``Schedule'' link in the
top right-hand corner. Now, you can browse my office hours or schedule an appointment by clicking the correct link and
then reserving an open time slot. 

\subsection*{Course Meeting Schedule}

Lecture, Discussion, Presentations, and Group Work: Tuesday and Thursday, 11:00 am -- 12:15 pm \\
Laboratory Session: Wednesday, 2:30 pm -- 4:20 pm \\
Final Examination: Friday, December 13, 2013 at 9:00 am

\subsection*{Course Catalogue Description}

\begin{quote}
A corequisite seminar for all students participating in the Internship Program.  Internship students enroll twice, once
prior to and once following the Internship. Computer Science 500 focuses on expectations and planning, leading to the
Internship Proposal required for all students planning an internship.  Computer Science 501 provides the opportunity for
students who have completed the Internship to prepare written and oral reports on the Internship experience. The
student, in consultation with the faculty of the Department of Computer Science, is responsible for arranging for an
acceptable internship.  The courses meet together weekly for one-half a semester. Credit: One semester hour for each
course. Prerequisites: Completion of at least two core courses.
\end{quote}

\subsection*{Course Objectives}

\subsection*{Performance Objectives}

\subsection*{Required Textbooks}

\noindent
Students who want to improve their technical writing skills may consult the following books.

\noindent{\em BUGS in Writing: A Guide to Debugging Your Prose}. Lyn Dupr\'e. Second Edition,  ISBN-10: 020137921X,
ISBN-13: 978-0201379211, 704 pages, 1998.

\noindent{\em Writing for Computer Science}.  Justin Zobel. Second Edition,  ISBN-10: 1852338024, ISBN-13:
978-1852338022, 270 pages, 2004.

\subsection*{Grading Policy}

The grade that a student receives in this class will be based on the
following categories. All percentages are approximate and it is possible
for the assigned percentages to change during the academic semester if a
need to do so presents itself. 

\begin{center}
\begin{tabular}{ll}
Class Participation and Instructor Meetings & 5\% \\
First Examination & 15\% \\
Second Examination & 15\% \\
Final Examination & 20\% \\
Laboratory and Homework Assignments & 30\% \\
Final Project & 15\%
\end{tabular}
\end{center}

Each of the above grading categories has the following definition:

\begin{itemize}

	\item {\em Class Participation and Instructor Meetings}: All students are required to actively participate during
		all of the class sessions. Your participation will take forms such as answering questions about the required
		reading assignments, asking constructive questions of your group members, giving presentations, and leading a
		discussion session. Furthermore, all students are required to meet with the course instructor during office
		hours for a total of sixty minutes during the Fall 2013 semester.  These meetings must be scheduled through the
		instructor's reservation system and documented on a meeting record that you submit on the day of the final
		examination. A student will receive an interim and final grade for this category.

	
\end{itemize}

\subsection*{Class Policies}

\subsubsection*{Disability Services}

The Americans with Disabilities Act (ADA) is a federal anti-discrimination statute that provides comprehensive civil
rights protection for persons with disabilities.  Among other things, this legislation requires all students with
disabilities be guaranteed a learning environment that provides for reasonable accommodation of their disabilities.
Students with disabilities who believe they may need accommodations in this class are encouraged to contact Disability
Services at 332-2898.  Disability Services is part of the Learning Commons and is located in Pelletier Library.
Please do this as soon as possible to ensure that approved accommodations are implemented in a timely fashion.

\subsubsection*{Honor Code}

The Academic Honor Program that governs the entire academic program at Allegheny College is described in the Allegheny
Course Catalogue.  The Honor Program applies to all work that is submitted for academic credit or to meet non-credit
requirements for graduation at Allegheny College.  This includes all work assigned for this class (e.g., examinations,
laboratory assignments, and the final project).  All students who have enrolled in the College will work under the Honor
Program.  Each student who has matriculated at the College has acknowledged the following pledge:

\begin{quote}
I hereby recognize and pledge to fulfill my responsibilities, as defined in the Honor Code, and to maintain the
integrity of both myself and the College community as a whole.  
\end{quote}

\end{document}
